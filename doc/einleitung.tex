% !TEX root =  master.tex
\chapter[Einleitung]{Einleitung{\hfill \normalsize Fabio Westphal}} 

Kurt Tucholsky schrieb bereits vor Hundert Jahren, was auch heute eine Herausforderung der Teamarbeit darstellt und worauf im späteren Verlauf näher eingegangen wird: 
\vspace*{\fill} 
\begin{quote} 
	\centering 
	\enquote{Einer hackt Holz, und dreiunddreißig stehen herum – die bilden die Zentrale.}
\end{quote}
\vspace*{\fill} 
Diese Seminararbeit begleitet das Projekt zur Erstellung einer Reservierungswebseite für Kinotickets. Dabei wird insbesondere die Vorgehensweise vom Entwurf bis zur Implementierung beschrieben. Es werden Hintergrundinformationen zu verwendeten Technologien und Methoden geliefert. Außerdem werden Einblicke in die Projektplanung, die Rahmenbedingungen und Zusammenarbeit im Team gegeben. \newline
Zunächst wird durch die Analyse einer bestehenden Lösung der Status Quo ermittelt und die Anforderungen in einer Soll-Analyse dargelegt. Als Abschluss dieser Phase steht eine User-Story-Map.
Anschließend werden verschiedene Entwurfsphasen beschrieben und die dazugehörigen Modelle erläutert. Beim Design-Entwurf wird eine erste visuelle Idee der Webseite dargestellt. Während des technischen Entwurfs kommen diverse Entwurfsprinzipien zum Tragen. Die Ergebnisse dieser Phase sind neben dynamsichen Modellen ein ER-Diagramm und ein Klassendiagramm.
Schließlich wird die Umsetzung des Entwurfs beschrieben. Dabei geht es zunächst um die verwendeten Werkzeuge und Entwicklungsumgebungen, ehe der Aufbau und die Konzepte des Backends erklärt werden. Analog werden beim Frontend die verwendeten Technologien sowie die Bewältigung der Anforderungen  -- und welche Rolle dabei die Benutzersicht spielt -- eruiert. Zudem wird beschrieben, auf welche Weise Back- und Frontend verknüpft wurden und welche Hindernisse es dabei gab. Letztlich wird durch Modultests die Qualität und Funktionsweise des Codes sichergestellt.
Darüber hinaus wird auf die Herausforderungen und Möglichkeiten einer realtiv großen Gruppe eingegangen. Die Ansätze, wie die Aufgaben bewältigt wurden, werden ebenso evaluiert wie die daraus generierten Learnings.
