% !TEX root =  master.tex
\chapter{Methodik}

In diesem Kapitel wird erläutert, welches Modell zur Softwareentwicklung als Vorbild genommen wurde, sowie in welcher Weise das Projektteam zusammengearbeitet hat. 

\section{Iteratives Modell}
Als Modell für den Entwicklungsprozess wurde von dem Team das iterative Modell gewählt. Aufgrund der Kürze des Entwicklungsprozesses hat sich das Team an dem Modell orientiert, es jedoch nicht komplett umsetzen können. Im Folgenden werden zunächst die einzelnen Bestandteile des Modells erklärt. Danach erfolgt eine Erläuterung über die Anwendung des Modells durch das Team.

\subsubsection{Theorie}
Das iterative Modell beinhaltet, wie in \textbf{Abbildung} ersichtlich,  die Bestandteile Anforderungsspezifikation, Entwurf, Implementierung und Test. Danach erfolgt die Freigabe der fertigen Software. 
Zu Beginn werden in der Anforderungsspezifikation die Anforderungen an das Softwareprodukt festgestellt. Dazu wird mithilfe der Analyse betrachtet, was das Programm später leisten soll. Als nächstes erfolgt der Entwurf, in dem ermittelt wird, wie die zuvor festgestellten Anforderungen umgesetzt werden können. Ergebnisse des Entwurfs können bereits erste objektorientierte Bestandteile sein, die für die Umsetzung benötigt werden. Wenn die Grundlagen für die Umsetzung, wie die verwendeten Programmiersprachen entschieden wurden, kann mit der Umsetzung begonnen werden. Im nachfolgenden Schritt erfolgt die Testphase, in der der geschriebene Programmcode auf Fehler untersucht wird. Des Weiteren werden relevante Testfälle ermittelt und durchgeführt. 

Die zuvor vorgestellten Bestandteile sind aus dem Wasserfallmodell abgeleitet. Der Unterschied der beiden Modelle besteht in der Abarbeitung der verschiedenen Phasen. Während in dem Vorgängermodell dem Wasserfallmodell jede Phase der Reihe nach durchlaufen wird, sind in dem iterativen Modell Rücksprünge in vorige Phasen möglich. Dadurch lassen sich während der Entwicklung auftretende Änderungen und neue Anforderungen noch in das Softwareprodukt einarbeiten. Somit können die Entwicklungsfortschritte dem Kunden vorgestellt und sein Feedback eingearbeitet werden. 

\subsubsection{Anwendung des Teams}
Das Projektteam hat für die zu entwickelnde Software zunächst aus den Projektvorgaben von Hr. Thielsch die Anforderungen herausgezogen und mit einer Analyse zu bereits bestehenden Lösungen eines Kinoreservierungssystems begonnen. Mithilfe von Personas und User-Story-Mapping wurden Personen entwickelt, die die fertige Software nutzen werden. Außerdem wurden in der Entwurfsphase überlegt, wie die festgestellten Anforderungen umgesetzt werden können. Zusätzlich wurde mit dem Entwurf eines Prototyps angefangen. 

Die bis dahin erfolgten Ergebnisse wurden dem Kunden präsentiert und Feedback eingeholt. Nachfolgend hat das Team mit der Implementierung des Produkts begonnen. Gleichzeitig erfolgte ein Rücksprung in die erste Phase, da das Feedback des Kunden eingearbeitet wurde. Dieses hat das Team außerdem dazu veranlasst eine erneute Ist-Analyse durchzuführen. Deren Ergebnisse wurden von dem Team eingearbeitet, wodurch einzelne Produkte des Analysephase angepasst wurden. 

Nach dem Start aber vor Fertigstellung der Implementierung wurden bereits erste Tests geschrieben um die Funktionalität der bis dahin erstellten Software sicherzustellen. Durch das Weiterentwickeln der Software und dem gleichzeitigen Schreiben von Tests erfolgt ein ständiger Sprung und Rücksprung zwischen den Phasen Implementierung und Test. Dies liegt vor allem an der Teamgröße, da voneinander unabhängige Entwicklungen gleichzeitig erfolgen können. 


\section{SCRUM}
Als Modell der Zusammenarbeit wurde SCRUM gewählt, da dieses das gleichzeitige Arbeiten an verschiedenen Phasen des iterativen Modells erlaubt und fördert. Außerdem hat SCRUM das Ziel ein agiles Umfeld für die Entwicklung und Auslieferung von Softwareprodukten zu schaffen. Da das iterative Modell ebenfalls eine agile Vorgehensweise beinhaltet, passen die gewählte Methode der Softwareentwicklung mit der Zusammenarbeit des Teams zusammen. 

\subsubsection{Theorie}
In SCRUM steht vor allem die Interaktion innerhalb des Teams sowie zu Projektbetroffenen im Vordergrund. Außerdem wird versucht innerhalb von kurzen Zeitabständen funktionierende Software ausliefern zu können und die Agilität des Prozesses schnell auf Änderungen reagieren zu können. 

Der Entwicklungsprozess wird in kleinere Abschnitte, sogenannte Sprints, unterteilt. Am Ende jedes Sprints ist das Ziel eine lauffähige Software vorweisen zu können. Zu Beginn eines Projekts werden alle bestehenden Anforderungen in dem Produkt Backlog festgehalten. In der Sprint Planung werden werden einzelne Anforderungen ausgewählt, um die das Produkt in dem nächsten Sprint erweitert werden soll. Diese Anforderungen werden im Sprint Backlog festgehalten. 

In einem Sprint kann sich jedes Teammitglied einen Teil einer Anforderung heraussuchen, die es bearbeiten möchte. Das Entwicklungsteam arbeitet also selbst-organisiert und agil. Täglich findet ein kurzes Meeting statt, welches zum Ziel hat die anderen Teammitglieder über den Fortschritt des letzten Tages zu informieren, wie die nächsten Aufgaben aussehen, sowie an welchen Stellen Probleme bestehen. Dadurch wird die Kommunikation des Teams gefördert und Probleme schnell aufgezeigt und können nach dem Meeting besprochen werden. 

\subsubsection{Umsetzung durch Team}