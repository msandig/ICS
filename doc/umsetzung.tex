% !TEX root =  master.tex
\chapter{Umsetzung} \label{umsetzung}
	\section{Organisation}
	GitHub
	\section{Backend}
	\section{Frontend}
	\section{Modultests}
	Wir testen die Besten.
	\section{Ausblick}
	\subsection{Lessons Learned}
	Die Bereiche, an denen gearbeitet wurde, waren vielfältig. Zuerst waren alle an der Analyse sowie der Erarbeitung des Entwurfs beteiligt. Dann musste sich das Team erarbeiten, wie ein solches Projekt umgesetzt werden kann. Die Programmierung des Backends war für einige der Teammitglieder komplett neu, aber da das Team sehr unterschiedliche Fähigkeiten besitzt, war dies in der geplanten Projektzeit möglich. Auch die Erstellung des Frontends war eine interessante Erfahrung. Die Verbindung von diesem mit dem Backend war für die meisten von uns eine weitere Herausforderung, da in diesem Bereich noch wenige Erfahrungen gemacht wurden. 
	Neben dem technischen Wissen haben die Teammitglieder auch einige weitere Kompetenzen in dem Projekt erworben. Eine davon ist, dass es sehr wichtig ist, andere Projekte, an denen die Teammitglieder außerhalb des Moduls arbeiten müssen, bei dem Projektplans zu berücksichtigen. Obwohl dies sehr schwierig ist, kann dies helfen, den Projektplan genauer und realistischer zu erstellen. Es ist sehr wichtig, für jede Aufgabe genügend Zeit einzuplanen, um unerwartete Herausforderungen zu bewältigen. Die selbständige Erarbeitung von Vorgehensweisen und Lösungskonzepte ist eine weitere in diesem Projekt erworbene Kompetenz.
	Eine weitere Hürde war die Gruppengröße von acht Personen. Es ist schwierig das Potenzial der Gruppe voll ausschöpfen, weil die Arbeit innerhalb der Gruppe gut organisiert werden muss.  Um eine übermäßige Koordination der Mitglieder und lange Kommunikationswege zu vermeiden, haben wir die Gruppe in Untergruppen mit verschiedenen Aufgabenbereichen aufgeteilt. Somit konnte eine schnelle Entscheidungsfindung und kurzfristige Absprachen garantiert werden. Insgesamt wurden in der engen Projektzeit viele unterschiedliche Kompetenzen erworben und vor allem die theoretischen Inhalte der Vorlesung Systemanalyse angewandt.
	