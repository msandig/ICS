% !TEX root =  master.tex
\chapter{Evaluation} \label{Evaluation}
	
	\section{Ausblick}
	%Von Jonas als Ausblick für Frontend 
		Das gesamte Projektteam ist nach dem ersten Release der Intelligent Cinema Suite sehr zufrieden. Das Frontend beinhaltet alle Funktionalitäten, die als Mindestziel gesetzt wurden. Der gesamte Durchlauf eines Reservierungsprozesses kann problemlos durchgeführt werden. Hierbei wird dem Besucher der Website eine anschauliche Oberfläche geboten, die einfach zu verstehen ist und auf unterschiedlichsten Endgeräten genutzt werden kann. Trotz dieses äußerst zufriedenstellenden Ergebnisses, gibt es natürlich auch noch einige verbesserungswürdige Stellen im Frontend. Diese konnten leider aufgrund der begrenzten Zeit des Projektes nicht mehr umgesetzt werden. Ziel muss es sein, einen großen Teil dieser neuen Funktionalitäten und auch der Designveränderungen im nächsten Release mit auszuliefern. Ursprünglich war zu Beginn des Projektes geplant, eine Anmeldung von Benutzern zu ermöglichen. Leider konnte dies in der Zeit nicht umgesetzt werden. Zudem kann die Responsiveness der Website an einigen Stellen noch optimiert werden. Für uns lag der Fokus letztlich eher darauf, einen gesamten Durchlauf der Reservierung zu ermöglichen. Dies nahm doch mehr Zeit in Anspruch, als zunächst gedacht. Eine Erweiterung im nächsten Release soll auch sein, dass die Kinokarten nicht nur reserviert werden können, sondern auch direkt mit unterschiedlichen Zahlungsmitteln gekauft werden können. Zudem plant das Team, die Browser-Kompatibilität der Website zu verbessern, da bisher zwar die meisten, aber nicht alle gängigen Internet-Browser die Website wie erwünscht darstellen. Das wichtigste Learning für unser Team ist, dass die Entwicklung eines ansprechenden Frontends viel Zeit erfordert. Es gibt viele Kleinigkeiten zu beachten, die Zeit in Anspruch nehmen. Ein Vorteil ist, dass das Ergebnis immer gleich betrachtet werden kann und somit Verbesserungsvorschläge schnell eingebaut werden können. Wir haben gelernt, dass auch externe Betrachter gut helfen können. Durch das Befragen von Außenstehenden des Projektes konnten gute Ideen eingebracht werden, die als Entwickler möglicherweise überhaupt nicht in Betracht gezogen worden wären.
	
	\section{Lessons Learned} \label{Ausblick}
	% Milena zu Projektumsetzung 
	% Fehlt LaTex Zusammenführung (inhaltlich sowie formell)
	Die Bereiche, an denen gearbeitet wurde, waren vielfältig. Zuerst waren alle an der Analyse sowie der Erarbeitung des Entwurfs beteiligt. Dann musste sich das Team erarbeiten, wie ein solches Projekt umgesetzt werden kann. Die Programmierung des Backends war für einige der Teammitglieder komplett neu, aber da das Team sehr unterschiedliche Fähigkeiten besitzt, war dies in der geplanten Projektzeit möglich. Auch die Erstellung des Frontends war eine interessante Erfahrung. Die Verbindung von diesem mit dem Backend war für die meisten von uns eine weitere Herausforderung, da in diesem Bereich noch wenige Erfahrungen gemacht wurden. 
	Neben dem technischen Wissen haben die Teammitglieder auch einige weitere Kompetenzen in dem Projekt erworben. Eine davon ist, dass es sehr wichtig ist, andere Projekte, an denen die Teammitglieder außerhalb des Moduls arbeiten müssen, bei dem Projektplans zu berücksichtigen. Obwohl dies sehr schwierig ist, kann dies helfen, den Projektplan genauer und realistischer zu erstellen. Es ist sehr wichtig, für jede Aufgabe genügend Zeit einzuplanen, um unerwartete Herausforderungen zu bewältigen. Die selbständige Erarbeitung von Vorgehensweisen und Lösungskonzepte ist eine weitere in diesem Projekt erworbene Kompetenz.
	Eine weitere Hürde war die Gruppengröße von acht Personen. Es ist schwierig das Potenzial der Gruppe voll ausschöpfen, weil die Arbeit innerhalb der Gruppe gut organisiert werden muss.  Um eine übermäßige Koordination der Mitglieder und lange Kommunikationswege zu vermeiden, haben wir die Gruppe in Untergruppen mit verschiedenen Aufgabenbereichen aufgeteilt. Somit konnte eine schnelle Entscheidungsfindung und kurzfristige Absprachen garantiert werden. Insgesamt wurden in der engen Projektzeit viele unterschiedliche Kompetenzen erworben und vor allem die theoretischen Inhalte der Vorlesung Systemanalyse angewandt.
	
	
	% Sicherheit der Daten?
	
	%Das Ergebnis des Projekts \ac{ICS} ist ein Prototyp, der bereits einen guten Status eines Reservierungstools darstellt. Für einen produktiven Einsatz könnte eine zusätzliche Admin-Modus nützlich sein, um neue Filme anzulegen sowie das Kino zu konfigurieren. --> Use Cases Kassandra und Walter 
	% noch effektiver gestaltet werden
	
	