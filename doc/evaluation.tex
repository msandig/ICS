% !TEX root =  master.tex
\chapter{Evaluation} \label{Evaluation}
	
		Für ein gutes Projekt ist zwar der Erfolg zum Stichtag ausschlaggebend, die Retrospektive sollte deswegen aber nicht weniger Aufmerksamkeit erlangen. Nur durch eine reflektierte Betrachtung kann sowohl Positives als auch Negatives beim nächsten Projekt hilfreich sein.
		
	\section[Bewertung der Entwicklung von Frontend und Backend]{Bewertung der Entwicklung von Frontend und Backend{\hfill \normalsize Fabio Westphal}}
		Das gesamte Projektteam ist nach dem ersten Release der Intelligent Cinema Suite sehr zufrieden. Sowohl Frontend als auch Backend beeinhalten alle Funktionalitäten, die als Mindestziel gesetzt wurden. Der gesamte Durchlauf eines Reservierungsprozesses kann problemlos durchgeführt werden. Darauf lag während der einzelnen Iterationen der Fokus. Dies nahm jedoch mehr Zeit in Anspruch, als zunächst gedacht. 
		Dem Besucher der Website wird eine anschauliche Oberfläche geboten, die einfach zu verstehen ist und auf unterschiedlichsten Endgeräten genutzt werden kann. Trotz dieses äußerst zufriedenstellenden Ergebnisses, gibt es natürlich auch noch einige verbesserungswürdige Stellen im Frontend. Diese konnten leider aufgrund der begrenzten Zeit des Projektes nicht mehr umgesetzt werden. Ziel muss es sein, einen großen Teil dieser neuen Funktionalitäten und auch der Designveränderungen im nächsten Release auszuliefern. Ursprünglich war zu Beginn des Projektes geplant, eine Anmeldung von Benutzern zu ermöglichen. Dies wurde wegen zeitlichen Begrenzungen zurückgestellt. Zudem kann die Responsiveness der Website an einigen Stellen noch optimiert werden. Hierfür ist es in CSS beispielsweise möglich, mit dem \texttt{@media}-Befehl Inhalte geziehlt für bestimmte Bildschirmgrößen zu optimieren. Als Erweiterung im nächsten Release ist auch angedacht, dass die Kinokarten nicht nur reserviert, sondern auch direkt mit unterschiedlichen Zahlungsmitteln gekauft werden können. Zudem plant das Team, die Browser-Kompatibilität der Website zu verbessern, da bisher zwar die meisten, aber nicht alle gängigen Internet-Browser die Website wie erwünscht darstellen. 
		Die oben auf der Webseite angesiedelte Suchleiste enthält zur Zeit noch keine Funktionalität. Somit wäre dieses Feature ebenfalls im nächsten Release anzugehen.
		
		Das wichtigste Learning in diesem Bereich war sicherlich, dass die Entwicklung eines ansprechenden Frontends viel Zeit erfordert. Es gibt viele Kleinigkeiten und Details zu beachten, die Zeit in Anspruch nehmen. Ein Vorteil ist, dass das Ergebnis von Änderungen immer sofort betrachtet werden kann und sich somit Verbesserungsvorschläge schnell einbauen lassen. Dabei können durchaus auch externe Betrachter - sei es aus anderen Gruppen oder unabhängige Personen - mit ihrer Meinung hilfreich sein. Durch das Befragen von Außenstehenden des Projektes konnten gute Ideen eingebracht werden, die als Entwickler möglicherweise überhaupt nicht in Betracht gezogen worden wären. Was ein Nutzer als intuitiv ansieht, spiegelt nicht zwangsläufig die Ideen des Entwicklers wieder. Beispielsweise stammt der Vorschlag, die Filmkacheln in der Übersicht mit einem Popup für Trailer und Informationen interaktiv zu gestalten, von einem außenstehenden Befragten. 
	
		Nicht minder zufrieden sind die Teammitglieder des Backends. Die Qualität des Codes befindet sich nach Meinung des Teams auf einem sehr guten Niveau, besonders angesichts des begrenzten Vorwissens. Auch hier funktionieren alle notwendigen Schnittstellen für das Frontend.
		Bereits im Kapitel \ref{umsetzung:backend} wurden einige Themen angesprochen, die Aufgrund von Zeitgründen nicht umgesetzt werden konnten. Beispielweise wurde ein vorübergehendes Benutzermanagement in einem dafür nicht vorgesehenen Manager implementiert. Des Weiteren sollte bei der nächsten Umsetzung des Backend mehr Wert auf das \ac{DDD} gelegt bzw. die Grundsubstanz dessen eingehalten werden. Das Spring-Framework bietet zahlreiche Möglichkeiten, das \ac{DDD} in verschiedenen Varianten umzusetzen, welche zu bei Beginn der Implementierung jedoch teilweise noch nicht bekannt waren. Die Testabdeckung könnte im Hinblick auf die faktische Prozentzahl der getesteten Zeilen höher sein. Allerdings muss auch hier bedacht werden, dass in der Summe über 100 einzelne Tests geschrieben und die wichtigsten Klassen mit einer deutlich höheren Abdeckung getestet wurden.
		\newpage
\section[Retrospektive der Analyse- und Entwurfsphasen]{Retrospektive der Analyse- und Entwurfsphasen\\{\hfill \normalsize Fabio Westphal}}
		Zum Abschluss des Releases lassen sich Aussagen über die Bedienung der User-Stories der Personas treffen. Der Walking Skeleton wurde umgesetzt, jedoch ohne Login zum Kundenkonto. Somit ist die Persona Gustav (siehe Abb. \ref{fig:useCaseGustav}) vollständig bedient. Ebenso erfüllt wurde die erste User-Story der Persona Franzi (siehe Abb. \ref{fig:useCaseFranzi}). Walter und Kassandra waren noch nicht für das erste Release relevant und wurden folglich nicht bedient.
		Für eine vollständige Ist-Analyse hätte man optimalerweise noch weitere bestehende Kinoreservierungssysteme heranziehen sollen. Davon wurde abgesehen, da der Fokus eher auf das grundlegende Verständnis des Aufbaus gelegt wurde.

		Bei der Entwicklung des technischen Entwurfs musste sich auf das ER-Diagramm, das Klassendiagramm und das Aktivitätendiagramm beschränkt werden. Es wäre sehr hilfreich gewesen, noch ein relationelles Datenbankmodell zu erstellen. Einige Probleme, die wegen fehlender Informationen während der Entwicklung am Backend teilweise aufgetreten sind, hätten dadurch eher verhindert werden können. Darüber hinaus ist bei einer Fortsetzung des Projekts sinnvoll, ein Sequenzdiagramm anzulegen. Allgemein lässt sich sagen, dass es sich bei der Entwicklung des technischen Entwurfs um einen komplexen Prozess handelt. Es war eine ausführliche Zusammenarbeit der verschiedenen Teammitglieder notwendig, da es verschiedene Möglichkeiten gibt, ein Softwaresystem zu modellieren, wobei nicht jede Lösung die in Kapitel \ref{chapter:technischer_entwurf} genannten Ziele Skalierbarkeit, Effizienz und Anpassbarkeit erfüllt.
		
	\section[Bewertung der Organisation]{Bewertung der Organisation{\hfill \normalsize Fabio Westphal}}
		Die Herausforderungen einer etwas größeren Gruppe wurden auch bei den gewählten Tools zur Organisation deutlich. Das Zusammenführen der aktuellsten Version der Seminararbeit warf zum Beispiel Probleme auf, wenn mehrere Personen an einem Kapitel arbeiteten. Github konnte die verschiedenen Commits teilweise nicht korrekt zusammenführen, da es sich bei LaTeX nicht um eine typische zeilenbasierte Sprache handelt. So mussten die neuen oder überarbeiteten Texte alle an eine Person geschickt werden, welche diese dann zusammensetzte. Beim nächsten Mal müsste im Vorhinein eine besser geeignete Software für kollaboratives Schreiben gewählt werden. Außerdem kommt es bei mehreren Autoren schnell zu Redundanzen. Bestimmte projektspezifische Sachverhalte werden zwar in mehreren Kapiteln behandelt, müssen aber für den Leser nur einmal ausführlich erklärt werden. Letzlich musste die Arbeit am Ende nochmal in Gänze gelesen werde, um solche inhaltlichen Dopplungen zu minimieren. Vermeiden lässt sich das mit angemessem Aufwand wohl nicht.
		Wie im Kapitel \ref{chapter:SCRUM} bereits erläutert wurden die Vorgaben von SCRUM nicht komplett umgesetzt. Bei täglicher Zusammenarbeit an dem Projekt wäre eine stärkere Orientierung am SCRUM-Ablauf definitiv möglich und nötig gewesen. Zudem wurde die Rolle des Product-Owners nicht im eigentlichen Sinne ausgefüllt, da beispielsweise der Backlog nicht von ihm, sondern von allen Teammitgliedern gemeinsam aufgestellt wurde. Bezüglich des Iterativen Modells lässt sich fesstellen, dass aufgrund des kurzen Entwicklungszeitraums relativ wenig Kundenfeedback eingeholt wurde, um die einzelnen Schritte zu validieren. Dadurch haben wahrscheinlich weniger Rücksprünge stattgefunden als in einem üblichen Projekt. 
		
		Die zeitlich Komponente wurde streckenweise etwas unterschätzt, hier müssten andere Vorlesungen und Prüfungsleistungen mit mehr Puffer einkalkuliert werden. Insgesamt wurde die Projektplanung aber ebenso gut umgesetzt und wie die Kommunikation innerhalb des Teams, da regelmäßige Absprachen und dauerhafte Kommunikationskanäle zu einer klaren Aufgabenverteilung geführt haben.
		
	
	\section[Lessons Learned]{Lessons Learned{\hfill \normalsize Milena Zahn}} \label{Ausblick} 
	% Milena zu Projektumsetzung 
	% Fehlt LaTex Zusammenführung (inhaltlich sowie formell)
	Die Bereiche, an denen gearbeitet wurde, waren vielfältig. Zunächst wurde an der Analyse und der Erarbeitung des Entwurfs gearbeitet. Danach musste das Team sich erarbeiten, wie ein solches Projekt umgesetzt werden kann und was die ersten Schritte für die Umsetzung einer eigenständigen Anwendung sind. Dies war für die meisten Teammitglieder eine komplett neue Erfahrung. Ebenfalls die Verknüpfung des Backends mit dem Frontend war eine herausfordernde Aufgabe, die jedoch durch die unterschiedlichen Kompetenzen des Teams gemeistert werden konnte. 
	
	Neben dem technischen Wissen haben die Teammitglieder auch weitere Kompetenzen in dem Projekt erworben. Eine davon ist, dass es sehr wichtig ist, andere Projekte, an denen die Teammitglieder außerhalb des Moduls arbeiten müssen, bei dem Projektplans zu berücksichtigen. Obwohl dies sehr schwierig ist, kann dies helfen den Projektplan genauer und realistischer zu erstellen. Es ist sehr wichtig für jede Aufgabe genügend Zeit einzuplanen, um unerwartete Probleme zu bewerkstelligen. Die selbständige Erarbeitung von Vorgehensweisen und Lösungskonzepte ist eine weitere in diesem Projekt erworbene und essentielle Kompetenz.
	
	Eine weitere Hürde war die Gruppengröße von acht Personen. Es ist eine große Herausforderung, das Potenzial der Gruppe voll ausschöpfen, weil die Arbeit innerhalb des Teams gut organisiert werden muss. 
	Wie schon das Zitat von Trocholsky aus der Einleitung andeutet, kann Überorganisation dazu führen, dass zu viele Teammitglieder - zumindest über eine zu lange Zeit - mit der Koordination beschäftigt sind und letztlich nur ein Bruchteil der Ressourcen für die eigentliche zu erledigende Arbeit aufgewandt werden kann.
	Um eine übermäßige Koordination der Mitglieder und lange Kommunikationswege zu vermeiden, wurde die Gruppe in Untergruppen mit verschiedenen Aufgabenbereichen aufgeteilt. Somit konnten schnelle Entscheidungsfindungen und kurzfristige Absprachen garantiert werden. 
	
	Die Schwierigkeiten der Gruppengröße wiederspiegelte sich zudem in der Erstellung und vor allem der Zusammenführung der Seminararbeit. Eine gute Struktur und eine klare Aufgabenverteilung sind hierbei sehr wichtig. Die Zusammenführung und das Korrekturlesen der Arbeit nahm in Anbetracht der Anzahl der Projektbeteiligten mehr Zeit in Anspruch als zunächst eingeplant.
	Insgesamt wurden in der Projektzeit viele unterschiedliche, sowohl technische als auch methodische, Kompetenzen erworben und die theoretischen Inhalte der Vorlesungen des zweiten Semesters praktisch angewandt.
	
	
	
	