%
%   Prof. Dr. Julian Reichwald
%   auf Basis einer Vorlage von Prof. Dr. Jörg Baumgart
%   DHBW Mannheim
%
%
%	ACHTUNG: Für das Erstellen des Literaturverzeichnisses wird das modernere Paket biblatex
%			 in Kombination mit biber verwendet -- nicht mehr das ältere BibTex!
% 			 Bitte stellen Sie ggf. Ihre TeX-Umgebung
% 			 entsprechend ein (z.B. TeXStudio: Einstellungen --> Erzeugen --> Standard Bibliographieprogramm: biber)
%

\documentclass[
	12pt,
	BCOR=5mm,
	DIV=12,
	headinclude=on,
	footinclude=off,
	parskip=half,
	bibliography=totoc,
	listof=entryprefix,
	toc=listof,
	pointlessnumbers,
	plainfootsepline]{scrreprt}

%	Konfigurationsdatei einziehen
\input{config}

\usepackage[T1]{fontenc}
\usepackage{inconsolata}

\usepackage{color}

\definecolor{pblue}{rgb}{0.13,0.13,1}
\definecolor{pgreen}{rgb}{0,0.5,0}
\definecolor{pred}{rgb}{0.9,0,0}
\definecolor{pgrey}{rgb}{0.46,0.45,0.48}

\usepackage{listings}

\lstset{language=Java,
    showspaces=false,
    showtabs=false,
    breaklines=true,
    showstringspaces=false,
    breakatwhitespace=true,
    commentstyle=\color{pgreen},
    keywordstyle=\color{pblue},
    stringstyle=\color{pred},
    basicstyle=\ttfamily,
    moredelim=[il][\textcolor{pgrey}]{$$},
    %    morecomment=[s][\color{yellow}]{@}{\ },
   moredelim=[is][\textcolor{darkorange}]{\%\%}{\%\%},
    basicstyle=\ttfamily\fontsize{10}{12}\selectfont
}

\begin{document}

%% BITTE GEBEN SIE HIER DEN TITEL UND DIE AUTORIN / DEN AUTOR DER ARBEIT AN!
%% DIESE INFORMATIONEN _MÜSSEN_ GESETZT SEIN, UM TITELBLATT, ABSTRACT UND
%% EIGENSTÄNDIGKEITSERKLÄRUNG AUTOMATISCH ANZUPASSEN!
%\TitelDerArbeit{Visualisierung von Daten des Internet of Things mit besonderer Betrachtung der User Experience für den Verbraucher}
\TitelDerArbeit{Intelligent Cinema Suite}
\Firma{SAP SE}
\Kurs{WWI 17 SE B}

\begin{titlepage}
\begin{minipage}{\textwidth}
		\vspace{-2cm}
		\noindent \includegraphics[scale=0.28]{img/firmenlogo.jpg} \hfill   \includegraphics{img/logo.jpg}
\end{minipage}
\vspace{1.5em}
\sffamily
\begin{center}
	\textsf{\large{}Duale Hochschule Baden-W\"urttemberg\\[1.5mm] Mannheim}\\[2em]
	\textsf{\textbf{UMWI/Fallstudie: Prototyp Online-Reservierung Kino-Tickets }}\\[5mm]
	\textsf{\textbf{\Large{}\DerTitelDerArbeit}} \\[1.5cm]
	\textsf{Studiengang Wirtschaftsinformatik\\ Software Engineering}
	
	\vspace{2em}
	%\textsf{\Large{Sperrvermerk}}
\vfill

\begin{minipage}{\textwidth}

\begin{tabbing}
	Wissenschaftlicher Betreuer: \hspace{0.35cm}\=\kill
	Verfasser: 
		\> Dutzi Jonas, 6681598 \\
		\> Keller Sandra, 6046520 \\
		\> Köhler Dennis, 8900967 \\
		\> Sandig Martin, 8857640 \\
		\> Waage Felix, 3459154 \\
		\> Werner Yvonne, 8519757 \\
		\> Westphal Fabio, 6198411 \\
		\> Zahn Milena, 7488221 \\\\
	Firma: \> \DerNameDerFirma  \\[1.5mm]
	Kurs: \> \DieKursbezeichnung \\[1.5mm]
	Dozent: \> Gregor Tielsch \\[1.5mm]
	Abgabe: \> 11. Februar 2019
\end{tabbing}
\end{minipage}

\end{center}

\end{titlepage}

\pagenumbering{roman} % Römische Seitennummerierung
\normalfont

%--------------------------------
% Verzeichnisse - nicht benötige Verzeichnisse bitte auskommentieren / löschen.
%--------------------------------

%   Sperrvermerk
%\input{nondisclosurenotice}

%	Kurzfassung
%\input{abstract}

%	Inhaltsverzeichnis
\begin{singlespacing}
\tableofcontents
\end{singlespacing}
 
%	Abbildungsverzeichnis
\listoffigures

%	Tabellenverzeichnis
%\listoftables

%	Listingsverzeichnis
%\lstlistoflistings

\noindent\begin{minipage}{\textwidth}
\listoftables
\lstlistoflistings
\end{minipage}

% 	Algorithmenverzeichnis
%\listofalgorithms

% 	Abkürzungsverzeichnis (siehe Datei acronyms.tex!)
\clearpage
\chapter*{Abkürzungsverzeichnis}	
\addcontentsline{toc}{chapter}{Abkürzungsverzeichnis}

\begin{acronym}[RDBMS]
	\acro{CRUD}{Create, Read, Update, Delete}
	\acro{CSS}{Cascading Style Sheet}
	\acro{ICS}{Intelligent Cinema Suite}
	\acro{DAO}{Data Access Object}
	\acro{DDD}{Domain Driven Design}
	\acro{DI}{Dependency Injection}
	\acro{GUI}{Graphical User Interface}
	\acro{HTML}{Hypertext Markup Language}
	\acro{HTTP}{Hypertext Transfer Protocol}
	\acro{ID}{Identifier}
	\acro{IoC}{Inversion of Control}
	\acro{JDBC}{Java Database Connectivity}
	\acro{JSON}{JavaScript Object Notation}
	\acro{POM}{Project Object Model}
	\acro{REST}{Representational State Transfer}
	\acro{URL}{Uniform Resource Locator}
	\acro{UUID}{Universally Unique Identifier}
	\acro{WWW}{World Wide Web}
\end{acronym}

\ohead{Acronyms} % Neue Header-Definition

%--------------------------------
% Start des Textteils der Arbeit
%--------------------------------
\clearpage
\ihead{\chaptername~\thechapter} % Neue Header-Definition (inner header)
\ohead{\headmark} % Neue Header-Definition (outer header)
\pagenumbering{arabic}  % Arabische Seitenzahlen

%	Anleitungs-Datei anleitung.tex einziehen. Auf diese Weise sollten Sie versuchen, für jedes einzelne
% Kapitel eine eigene Datei anzulegen und mittels input-Kommando einzuziehen.
% !TEX root =  master.tex
\chapter{Einleitung}

In der Kürze liegt die Würze.
	



%% !TEX root =  master.tex
\chapter{Theoretische Grundlagen} 

	(OPTIONAL - vielleicht als Platzfüller)
% !TEX root =  master.tex
\chapter{Projektplanung}
	
	Unseren nicht vorhandener Plan erläutern mit Deadlines etc.
	\section{Organisation}
	\section{Projektplan}
	
% !TEX root =  master.tex
\chapter{Methodik}

\section[Iteratives Modell]{Iteratives Modell {\hfill \normalsize Yvonne Werner}}
Als Modell für den Entwicklungsprozess wurde von dem Team das iterative Modell gewählt. Aufgrund der Kürze des Entwicklungsprozesses hat sich das Team an dem Modell orientiert, es jedoch nicht komplett umsetzen können. Im Folgenden werden zunächst die einzelnen Bestandteile des Modells erklärt. Danach erfolgt eine Erläuterung über die Anwendung des Modells durch das Team.

\subsubsection{Theorie}
Das iterative Modell beinhaltet, wie in Abbildung \vref{fig:iteratModell} ersichtlich,  die Bestandteile Anforderungsspezifikation, Entwurf, Implementierung und Test. Danach erfolgt die Freigabe der fertigen Software. 
Zu Beginn werden in der Anforderungsspezifikation die Anforderungen an das Softwareprodukt festgestellt. Dazu wird mithilfe der Analyse betrachtet, was das Programm später leisten soll. Als nächstes erfolgt der Entwurf, in dem ermittelt wird, wie die zuvor festgestellten Anforderungen umgesetzt werden können. Ergebnisse des Entwurfs können bereits erste objektorientierte Bestandteile sein, die für die Umsetzung benötigt werden. Wenn die Grundlagen für die Umsetzung, wie die zu verwendende Programmiersprachen entschieden wurden, kann mit der Umsetzung begonnen werden. Im nachfolgenden Schritt erfolgt die Testphase, in der der geschriebene Programmcode auf Fehler untersucht wird. Des Weiteren werden relevante Testfälle ermittelt und durchgeführt. 
\begin{figure}[H]
	\centering 
	\includegraphics[width=14cm]{img/iterativesModell.png}
	\captionsetup{format=hang}
	\centering\caption[Iteratives Modell]{\label{fig:iteratModell}Iteratives Modell \\Quelle: Skript Systemanalyse S.25, Gregor Tielsch}
\end{figure}

Die zuvor vorgestellten Bestandteile sind aus dem Wasserfallmodell abgeleitet. Der Unterschied der beiden Modelle besteht in der Abarbeitung der verschiedenen Phasen. Während in dem Vorgängermodell, dem Wasserfallmodell, jede Phase der Reihe nach durchlaufen wird, sind in dem iterativen Modell Rücksprünge in vorige Phasen möglich. Dadurch lassen sich während der Entwicklung auftretende Änderungen und neue Anforderungen noch in das Softwareprodukt einarbeiten. Somit können die Entwicklungsfortschritte dem Kunden vorgestellt und sein Feedback eingearbeitet werden. 

\subsubsection{Anwendung des Teams}
Das Projektteam hat für die zu entwickelnde Software zunächst aus den Projektvorgaben von Hr. Tielsch die Anforderungen herausgezogen und mit einer Analyse zu bereits bestehenden Lösungen eines Kinoreservierungssystems begonnen. Mithilfe von Personas und User-Story-Mapping wurden typische Endnutzer modelliert, die die fertige Software nutzen werden. Außerdem wurde in der Entwurfsphase überlegt, wie die festgestellten Anforderungen umgesetzt werden können. Zusätzlich wurde mit dem Entwurf eines Prototyps angefangen. 

Die bis dahin erfolgten Ergebnisse wurden dem Kunden präsentiert und Feedback eingeholt. Nachfolgend hat das Team mit der Implementierung des Produkts begonnen. Gleichzeitig erfolgte ein Rücksprung in die erste Phase, um das Feedback des Kunden einarbeiten zu können. Dieses hat das Team außerdem dazu veranlasst eine erneute Ist-Analyse durchzuführen. Deren Ergebnisse wurden von dem Team umgesetzt, wodurch einzelne Produkte des Analysephase angepasst wurden. 

Nach dem Start aber vor Fertigstellung der Implementierung wurden bereits erste Tests geschrieben um die Funktionalität der bis dahin erstellten Software sicherzustellen. Durch das Weiterentwickeln der Software und dem gleichzeitigen Schreiben von Tests erfolgt ein ständiger Wechsel zwischen den Phasen Implementierung und Test. Dies liegt vor allem an der Teamgröße, da voneinander unabhängige Entwicklungen von den vielen Mitgliedern gleichzeitig erfolgen können. 
 
\section[SCRUM]{SCRUM{\hfill \normalsize Yvonne Werner}} \label{chapter:SCRUM}
Als Modell der Zusammenarbeit wurde Scrum gewählt, da dieses das gleichzeitige Arbeiten an verschiedenen Phasen des iterativen Modells erlaubt und fördert. Außerdem hat Scrum das Ziel ein agiles Umfeld für die Entwicklung und Auslieferung von Softwareprodukten zu schaffen. Da das iterative Modell ebenfalls eine agile Vorgehensweise darstellt, passen die gewählte Methode der Softwareentwicklung mit der Zusammenarbeit des Teams zusammen. 

\subsubsection{Theorie}
In Scrum steht vor allem die Interaktion innerhalb des Teams sowie zu Projektbetroffenen im Vordergrund. Außerdem wird versucht innerhalb von kurzen Zeitabständen funktionierende Software ausliefern zu können und durch die Agilität des Prozesses schnell auf Änderungen reagieren zu können. 

Zu Beginn eines Projekts werden alle bestehenden Anforderungen in dem Produkt Backlog festgehalten. Wie in Abbildung \vref{fig:scrum} ersichtlich ist, wird der Entwicklungsprozess für ein Produkt in kleinere Abschnitte, sogenannte Sprints, unterteilt. Die Sprint Planung, welche zu Beginn eines jeden Sprints stattfindet, dient der Auswahl von einzelnen Anforderungen, um welche das Produkt in dem nächsten Sprint erweitert werden soll. Zum Festhalten und Spezifizieren der gewählten Anforderungen dient der Sprint Backlog. Am Ende jedes Sprints ist das Ziel lauffähige Software vorweisen zu können. 

In einem Sprint kann sich dann jedes Teammitglied eine Teilaufgabe einer Anforderung heraussuchen, die es bearbeiten möchte. Das Entwicklungsteam arbeitet demnach selbst-organisiert und agil. Täglich findet ein kurzes Meeting statt, welches zum Ziel hat die anderen Mitglieder des Teams über den Fortschritt des letzten Tages, die nächsten Arbeitsschritte sowie Probleme zu informieren. Dadurch wird die Kommunikation des Teams gefördert und Probleme schnell aufgezeigt. Diese können nach dem Daily Scrum Meeting besprochen werden. 

\begin{figure}[H]
	\centering 
	\includegraphics[width=14cm]{img/Scrum.png}
	\captionsetup{format=hang}
	\centering\caption[Iteratives Modell]{\label{fig:scrum}SCRUM \footnotemark}
\end{figure}
\footnotetext{https://www.etventure.de/blog/digitallearning-2-agiles-arbeiten-mit-scrum/}

Neben dem Team, welches die Entwicklung vornimmt, existieren weitere wichtige Rollen in Scrum. Der Product Owner ist für das Produkt verantwortlich und entscheidet durch Kundenkontakt, welche Funktionalitäten das Produkt erhalten soll und in welcher Reihenfolge. Der Scrum Master achtet darauf, dass die Regeln für Scrum eingehalten werden und hält Probleme von dem Team fern, damit es sich auf die Weiterentwicklung konzentrieren kann.  

\subsubsection{Umsetzung durch Team}
Die Rolle des Scrum Masters hat Fabio Westphal übernommen. Er hat mit dem Management des Teams auf regelmäßigen Kontakt des gesamten Teams geachtet. Außerdem hat er bei Fragen sich darum bemüht diese zeitnah zu klären. Die Rolle des Kunden hat Gregor Tielsch dargestellt, welcher die Projektdefinition sowie die Anforderungen gestellt hat. Dadurch hat er ebenfalls Teile des Product Owners übernommen, da er den Rahmen des Projekts eingeschränkt hat. Allerdings wurde von dem Team besprochen, welche Funktionalitäten zuerst in das Produkt eingebaut werden sollen. Außerdem sind durch die Bewertung und Gemeinschaftsarbeit des Projekts alle Projektbeteiligten für das Produkt verantwortlich. Deshalb kann die Rolle des Product Owners keiner Person eindeutig zugeordnet werden.

Die Länge eines Sprints betrug, wie auch in Abbildung \vref{fig:projektplan} ersichtlich ist, ungefähr drei Wochen. Aufgrund von anderen Lehrveranstaltungen fand keine tägliche Zusammenarbeit für das Projekt statt, weshalb die Notwendigkeit von täglichen Meetings nicht bestand. Um dennoch einen kontinuierlichen Austausch ermöglichen zu können, wurde Slack als Kommunikationsmittel verwendet. Alternativ zu einem Daily Scrum Meeting wurde ein Jour Fixe eingeführt, damit sich das Team in regelmäßigen Abständen (einmal wöchentlich) über den aktuellen Stand des Produkts informieren kann. 

Die Entwicklung durch das Team wurde teilweise mittels Pair-Programming durchgeführt. Dieses bildet eine agile Methode zur Entwicklung, bei der zwei gleichberechtigte Entwickler an einem Computer arbeiten. Dabei schreibt ein Entwickler den Code, während ein zweiter diesen auf Korrektheit überprüft. Beide Entwickler sind gleichberechtigt, was dazu führt, dass in regelmäßigen Abständen die Rollen getauscht werden.
% !TEX root =  master.tex
\chapter{Analyse} \label{analyse}
	
	Analyse für den Durchblick.
	Studium, Modul \enquote{Fallstudie} et cetera
	
	\section{Anforderungsanalyse}
	
	\section{Ist-Analyse}
	Bestandsaufnahme existierender Reservierungssysteme
	
	\section{Soll-Analyse}
		
	 	\subsection{User-Research-Prozess} 
	 	User Research ist eine systematische Untersuchung der Ziele, Bedürfnisse und Fähigkeiten der Benutzer\autocite[Vgl.][S. 6]{Schumacher.2010}.
	 	Durch diesen Prozess lässt sich sicherstellen, dass die entwickelte Software den Nutzern einen Mehrwert bietet und an deren Bedürfnisse angepasst ist.
		Der Prozess startet mit der Definition der \textit{User Profile}. Diese klassifizieren Endanwender mit Nutzerprofilen, damit eine konkrete Vorstellung der Anwender geschaffen wird. Anhand von detaillierten Beschreibungen von Attributen lässt sich die Nutzergruppe identifizieren.
		
		\begin{table}[H]
			\centering
			\begin{tabular}{p{5,5cm} || c | c }
				\textbf{Attribute} & \textbf{Endanwender} & \textbf{Mitarbeiter} \\\toprule
				Alter &  16 - 99 Jahre &  16 - 99 Jahre \\
				Geschlecht &  männlich und weiblich &  männlich und weiblich  \\
				Medienkompetenz &  ja und nein &  ja  \\
				Erfahrung Onlinereservierung &  ja und nein &  ja  \\
			\end{tabular}
			\caption[User Profile]{\label{tab:tabelleUserProfile}User Profile}
		\end{table}
		
		Im nächsten Schritt werden Personas aus den User Profiles erstellt. Personas sind fiktive Personen, die für eine Nutzergruppe steht. Der Zweck von Personas ist, dass Entwickler Empathie und Einfühlungsvermögen zu den konkreten Anwendern aufbauen können. 
		
		\begin{figure}[H]
			\subfigure[Persona Franzi]{\includegraphics[width=0.49\textwidth]{img/franzi.png}} 
			\subfigure[Persona Gustav]{\includegraphics[width=0.49\textwidth]{img/gustav.png}} 
			\subfigure[Persona Kassandra]{\includegraphics[width=0.49\textwidth]{img/kassandra.png}} 
			\subfigure[Persona Walter]{\includegraphics[width=0.49\textwidth]{img/walter.png}} 
			\caption[Personas ]{\label{fig:personas}Personas [DRAFT - detaillierter!] }
		\end{figure} 
		
		Darauf aufbauend werden Use Cases für die Personas entwickelt. Diese beschreiben Szenarien, wie die Personas das Endprodukt verwenden werden und welche Vorgehensweisen und Anforderungen dabei haben. 
		
		\begin{figure}[H]
			\begin{tabular}{p{13cm}}
				\textbf{Franzi Regard} \\\toprule
				Als Franzi möchte ich mich schnell vom meinem Smartphone oder Laptop aus über das aktuelle Kino-Programm informieren und Kino-Tickets reservieren und kaufen, um mit meinen Freunden einen Film anzuschauen. Dabei lege ich viel Wert auf ansprechendes Design und eine schnelle Abwicklung.
			\end{tabular}
			\caption[Use Case Franzi]{\label{fig:useCaseFranzi} Use Case Franzi [DRAFT]}
		\end{figure}
	
		\begin{figure}[H]
			\begin{tabular}{p{13cm}}
				\textbf{Gustav Gast} \\\toprule
				Als Gustav möchte ich Kino-Tickets reservieren, um mit meiner Familie einen Familienabend zu verbringen. Eine intuitive Anwendung ist für mich sehr wichtig, weil noch nie online Tickets reserviert habe. 
			\end{tabular}
			\caption[Use Case Gustav]{\label{fig:useCaseGustav} Use Case Gustav [DRAFT]}
		\end{figure}
	
		\begin{figure}[H]
			\begin{tabular}{p{13cm}}
				\textbf{Kassandra Caisse} \\\toprule
				Als Kassandra möchte ich mir die Reservierungen anschauen und Reservierungen anpassen können. 
			\end{tabular}
			\caption[Use Case Kassandra]{\label{fig:useCaseKassandra} Use Case Kassandra [DRAFT]}
		\end{figure}

		\begin{figure}[H]
			\begin{tabular}{p{13cm}}
				\textbf{Walter Gerer} \\\toprule
				Als Walter möchte ich das Kino-Programm verwalten und einpflegen. 
			\end{tabular}
			\caption[Use Case Walter]{\label{fig:useCaseWalter} Use Case Walter [DRAFT]}
		\end{figure}
		
		Schließlich wird aus den Personas eine User-Story-Map erstellt. In dieser werden die daraus hervorgehenden Features visuell geplant und in unterschiedliche Releases priorisiert. 
		
	\begin{center}
		{\textbf{[User-Story-Map // Größe?]}}
	\end{center}
		
	 	
%		\subsection{Vorgabe}
%		\subsection{Rahmenbedingungen}
%		\subsection{Bewertungskriterien}
% !TEX root =  master.tex
\chapter{Entwurf}\label{entwurf}
	\section[Design Entwurf]{Design Entwurf {\hfill \normalsize Milena Zahn}}\label{design}
		\begin{figure}[H]
			\centering 
			\includegraphics[width=14cm]{img/mockUp1.png}
			\captionsetup{format=hang}
			\caption[Mockup Startseite]{\label{fig:mockUpStartseite} Mockup Startseite }
		\end{figure}
		\begin{figure}[H]
			\centering 
			\includegraphics[width=14cm]{img/mockUp2.png}
			\captionsetup{format=hang}
			\caption[Mockup Programm]{\label{fig:mockUpProgramm} Mockup Programm }
		\end{figure}
		\begin{figure}[H]
			\centering 
			\includegraphics[width=14cm]{img/mockUp3.png}
			\captionsetup{format=hang}
			\caption[Mockup Sitzplan]{\label{fig:mockUpSitzplan} Mockup Sitzplan }
		\end{figure}
	
	\section[Technischer Entwurf]{Technischer Entwurf{\hfill \normalsize Felix Waage}} 	

		Das \ac{ICS} ist ein komplexes Softwaresystem, welches aus verschiedenen Komponenten, Services und Schichten besteht. Aus diesem Grund war es bereits von Beginn an wichtig einen genauen Entwurf der späteren Softwarearchitektur zu entwickeln. Diese Vorgehensweise hat nicht nur den Vorteil, dass Zusammenhänge besser verstanden werden, sondern Probleme auch schneller zu finden und zu beheben sind. Darüber hinaus wird nur so eine effektive und gute Zusammenarbeit zwischen den Teammitgliedern ermöglicht und später die Wartung des gesamten Softwaresystem vereinfacht.
		Im weiteren Verlauf dieses Kapitel sollen die verschiedenen Schichten erläutert und deren Kommunikation untereinander beschrieben werden. Des Weiteren wird das Datenbankmodell definiert und näher auf die Architektur des Backends eingegangen.
		
		\subsection{Entwurfsprinzipien}
		Beim Entwurf eines Softwaresystem ist es besonders wichtig konsistent und gründlich zu arbeiten. Dies ist bedingt durch die Tatsache, dass immer mehrere Personen am \ac{ICS} arbeiten und den Entwurf verstehen müssen. Aus diesem Grund wurden vor Beginn der Entwicklung des Entwurfs einige Prinzipien festgelegt. Alle Prinzipien, welche folgend aufgelistet und erläutert werden, sollen unter anderem die Übersichtlichkeit, die Wartbarkeit und die Wiederverwendbarkeit des gesamten Projekts oder von Teilen davon ermöglichen.
		\begin{itemize}
			\item \textbf{Das Prinzip einer einzigen Verantwortung} -- Um die Komplexität und Organisation des Softwareprojekts beherschen zu können, wird das Projekt in verschiedene Module aufgeteilt. Dabei könne einzelene Module wieder aus anderen Modulen zusammen gesetzt sein. Es gilt so Komplexitäten aufzulösen. Jedes Modul übernimmt dabei genau eine Verantwortung und jede Verantwortung wir von genau einem Modul übernommen. Verantwortung ist in diesem Fall die Verpflichtung eine Anforderung umzusetzen. \autocite[Vgl.][]{Lahres.2015}
			\item \textbf{Trennung der Anliegen} -- Jedes Anliegen in einer Anwendung soll durch eigenes Modul realisiert werden. Ein mögliches Anliege wäre zum Beispiel die Transaktionssicherheit, welche unter anderem bei der Reservierung benötigt wird, jedoch auch bei weiteren Anforderung wiederverwendet werden können soll.\autocite[Vgl.][]{Lahres.2015} 
			\item \textbf{Wiederholungen vermeiden} -- Wenn gleiche Funktionalitäten in einem Softwaresystem mehrfach verwendet werden, sollte diese in ein Modul ausgelagert werden, um mögliche Redundanzen zu vermeiden. Dies könnte vor allem dann zum Problem führen, wenn im Code Fehler entdeckt wurden und dieser Fehler so an mehreren Stellen im Quelltext behoben werden muss. Dies stellt eine große Fehlerquelle dar und sollte somit vermieden werden.\autocite[Vgl.][]{Lahres.2015} 
			\item \textbf{Trennung der Schnittstelle von der Implementierung} -- Jedes Modul sollte nur von einer klar definierten Schnittstelle von einem anderen Modul abhängig sein. Dabei spielt die Implementierung der einzelnen Funktionalitäten keine Rolle. Der Quelltext der einzelnen Funktionalitäten soll demnach ausgetauscht werden können, ohne Änderungen an den Schnittstellenaufrufen vornehmen zu müssen. Dies macht das Softwaresystem verständlicher und einfacher zu warten.\autocite[Vgl.][]{Lahres.2015} 
			\item \textbf{Testbarkeit} -- Um direkt während der Entwicklung auf Fehler reagieren zu können ist es wichtig darauf zu achten, dass sich die einzelnen Module und Softwarekomponenten einzelnen Testen lassen. So werden neben der eigentlichen Funktionalität auch Unit-Test implementiert. Dies soll möglichst parallel zur Entwicklung der Funktionalität geschehen und muss beim Erstellen des Entwurfs beachtet werden.\autocite[Vgl.][]{Lahres.2015} 
		\end{itemize} 
		Wie in vermutlich jedem großen Softwareprojekt, kann es zu Sonderfällen kommen, wodurch nicht immer alle Prinzipien genau angewendet wurden. Im weiteren Verlauf dieses Kapitel wird an geeigneten Stellen noch einmal auf verschieden Prinzipien verwiesen, um deren Anwendung besser zu erläutern.
		\subsection{Schichtenmodell ICS}
		\begin{figure}[H]
			\centering 
			\includegraphics[width=14cm]{img/Schichtenmodell_ICS.pdf}
			\captionsetup{format=hang}
			\caption[Klassendiagramm]{\label{fig:Schichtenmodell} Schichtenmodell des ICS }
		\end{figure}
		Die Entwicklung des technischen Entwurfs für das \ac{ICS} wurde damit begonnen die notwendigen Schichten zu identifizieren und in Relation zueinander zu setzen. Es wurde beschlossen das Softwaresystem in drei Schichten aufzutrennen. 
		Die oberste Schicht ist das \glqq \textbf{Frontend}\grqq{}, welches die grafische Schnittstelle zum Benutzer darstellt. Über das Frontend kann der Benutzer zum Beispiel Filme suchen, Informationen zu Filmen einsehen und Tickets für eine Vorstellung reservieren. Diese Informationen erhält das Frontend durch HTTP-Request vom Backend.
		Im \glqq \textbf{Backend}\grqq{} ist die Fachlogik des Softwaresystem abgebildet, welche zum Beispiel zur Überprüfung der Korrektheit einer Reservierung benötigt wird. Darüber hinaus werden vom Backend die benötigten Restschnittstellen bereitgestellt und die Verbindung zur Datenbank organisiert. 
		Die \textbf{Dankbank} dient der Speicherung sämtlicher Daten und Informationen. Sie ist direkt mit dem Backend verbunden und nimmt Anfragen über SQL entgegen. 
		Für die Trennung des Softwaresystems in drei Schichten gibt es verschieden Gründe. Zum einen werden für die Implementierung des Frontends andere Technologien verwendet als für das Backend oder der Datenbank. Darüber hinaus verlangen die Anforderungen an das \ac{ICS} eine zentrale Datenhaltung, was sich am besten durch unterschiedliche Schichten realisieren lässt.
		\subsection{Statische Modelle}
		\begin{figure}[H]
			\centering 
			\includegraphics[width=14cm]{img/klassendiagramm.png}
			\captionsetup{format=hang}
			\caption[Klassendiagramm]{\label{fig:klassendiagramm} Klassendiagramm }
		\end{figure}
		
		\begin{figure}[H]
			\centering 
			\includegraphics[width=15cm]{img/erModell.png}
			\captionsetup{format=hang}
			\caption[Entity Relationship Datenmodell]{\label{fig:erModell} Entity Relationship Datenmodell}
		\end{figure}
		
		\subsection{Dynamische Modelle}
		\begin{figure}[H]
			\centering 
			\includegraphics[width=15cm]{img/adSeitenaufruf.pdf}
			\captionsetup{format=hang}
			\caption[Aktivitätsdiagramm Seitenaufruf]{\label{fig:aktivitätSeitenaufruf} Aktivitätsdiagramm Seitenaufruf}
		\end{figure}
		
		%           \begin{figure}[H]
		%               \centering 
		%               \includegraphics[width=14cm]{img/adReservierung.pdf}
		%               \captionsetup{format=hang}
		%               \caption[Aktivitätsdiagramm Seitenaufruf]{\label{fig:aktivitätSeitenaufruf} Aktivitätsdiagramm Reservierung}
		%           \end{figure}
% !TEX root =  master.tex
\chapter{Umsetzung} 
	\section{Organisation}
	Slack, GitHub
	\section{Backend}
	\section{Frontend}
	\section{Ausblick}
% !TEX root =  master.tex
\chapter{Evaluation} \label{Evaluation}
	
		Für ein gutes Projekt ist zwar der Erfolg zum Stichtag ausschlaggebend, die Retrospektive sollte deswegen aber nicht weniger Aufmerksamkeit erlangen. Nur durch eine reflektierte Betrachtung kann sowohl Positives als auch Negatives beim nächsten Projekt hilfreich sein.
		
	\section[Bewertung der Entwicklung von Frontend und Backend]{Bewertung der Entwicklung von Frontend und Backend{\hfill \normalsize Fabio Westphal}}
		Das gesamte Projektteam ist nach dem ersten Release der Intelligent Cinema Suite sehr zufrieden. Sowohl Frontend als auch Backend beeinhalten alle Funktionalitäten, die als Mindestziel gesetzt wurden. Der gesamte Durchlauf eines Reservierungsprozesses kann problemlos durchgeführt werden. Darauf lag während der einzelnen Iterationen der Fokus. Dies nahm jedoch mehr Zeit in Anspruch, als zunächst gedacht. 
		Dem Besucher der Website wird eine anschauliche Oberfläche geboten, die einfach zu verstehen ist und auf unterschiedlichsten Endgeräten genutzt werden kann. Trotz dieses äußerst zufriedenstellenden Ergebnisses, gibt es natürlich auch noch einige verbesserungswürdige Stellen im Frontend. Diese konnten leider aufgrund der begrenzten Zeit des Projektes nicht mehr umgesetzt werden. Ziel muss es sein, einen großen Teil dieser neuen Funktionalitäten und auch der Designveränderungen im nächsten Release auszuliefern. Ursprünglich war zu Beginn des Projektes geplant, eine Anmeldung von Benutzern zu ermöglichen. Dies wurde wegen zeitlichen Begrenzungen zurückgestellt. Zudem kann die Responsiveness der Website an einigen Stellen noch optimiert werden. Hierfür ist es in CSS beispielsweise möglich, mit dem \texttt{@media}-Befehl Inhalte geziehlt für bestimmte Bildschirmgrößen zu optimieren. Als Erweiterung im nächsten Release ist auch angedacht, dass die Kinokarten nicht nur reserviert, sondern auch direkt mit unterschiedlichen Zahlungsmitteln gekauft werden können. Zudem plant das Team, die Browser-Kompatibilität der Website zu verbessern, da bisher zwar die meisten, aber nicht alle gängigen Internet-Browser die Website wie erwünscht darstellen. 
		Die oben auf der Webseite angesiedelte Suchleiste enthält zur Zeit noch keine Funktionalität. Somit wäre dieses Feature ebenfalls im nächsten Release anzugehen.
		
		Das wichtigste Learning in diesem Bereich war sicherlich, dass die Entwicklung eines ansprechenden Frontends viel Zeit erfordert. Es gibt viele Kleinigkeiten und Details zu beachten, die Zeit in Anspruch nehmen. Ein Vorteil ist, dass das Ergebnis von Änderungen immer sofort betrachtet werden kann und sich somit Verbesserungsvorschläge schnell einbauen lassen. Dabei können durchaus auch externe Betrachter - sei es aus anderen Gruppen oder unabhängige Personen - mit ihrer Meinung hilfreich sein. Durch das Befragen von Außenstehenden des Projektes konnten gute Ideen eingebracht werden, die als Entwickler möglicherweise überhaupt nicht in Betracht gezogen worden wären. Was ein Nutzer als intuitiv ansieht, spiegelt nicht zwangsläufig die Ideen des Entwicklers wieder. Beispielsweise stammt der Vorschlag, die Filmkacheln in der Übersicht mit einem Popup für Trailer und Informationen interaktiv zu gestalten, von einem außenstehenden Befragten. 
	
		Nicht minder zufrieden sind die Teammitglieder des Backends. Die Qualität des Codes befindet sich nach Meinung des Teams auf einem sehr guten Niveau, besonders angesichts des begrenzten Vorwissens. Auch hier funktionieren alle notwendigen Schnittstellen für das Frontend.
		Bereits im Kapitel \ref{umsetzung:backend} wurden einige Themen angesprochen, die Aufgrund von Zeitgründen nicht umgesetzt werden konnten. Beispielweise wurde ein vorübergehendes Benutzermanagement in einem dafür nicht vorgesehenen Manager implementiert. Des Weiteren sollte bei der nächsten Umsetzung des Backend mehr Wert auf das \ac{DDD} gelegt bzw. die Grundsubstanz dessen eingehalten werden. Das Spring-Framework bietet zahlreiche Möglichkeiten, das \ac{DDD} in verschiedenen Varianten umzusetzen, welche zu bei Beginn der Implementierung jedoch teilweise noch nicht bekannt waren. Die Testabdeckung könnte im Hinblick auf die faktische Prozentzahl der getesteten Zeilen höher sein. Allerdings muss auch hier bedacht werden, dass in der Summe über 100 einzelne Tests geschrieben und die wichtigsten Klassen mit einer deutlich höheren Abdeckung getestet wurden.
		\newpage
\section[Retrospektive der Analyse- und Entwurfsphasen]{Retrospektive der Analyse- und Entwurfsphasen\\{\hfill \normalsize Fabio Westphal}}
		Zum Abschluss des Releases lassen sich Aussagen über die Bedienung der User-Stories der Personas treffen. Der Walking Skeleton wurde umgesetzt, jedoch ohne Login zum Kundenkonto. Somit ist die Persona Gustav (siehe Abb. \ref{fig:useCaseGustav}) vollständig bedient. Ebenso erfüllt wurde die erste User-Story der Persona Franzi (siehe Abb. \ref{fig:useCaseFranzi}). Walter und Kassandra waren noch nicht für das erste Release relevant und wurden folglich nicht bedient.
		Für eine vollständige Ist-Analyse hätte man optimalerweise noch weitere bestehende Kinoreservierungssysteme heranziehen sollen. Davon wurde abgesehen, da der Fokus eher auf das grundlegende Verständnis des Aufbaus gelegt wurde.

		Bei der Entwicklung des technischen Entwurfs musste sich auf das ER-Diagramm, das Klassendiagramm und das Aktivitätendiagramm beschränkt werden. Es wäre sehr hilfreich gewesen, noch ein relationelles Datenbankmodell zu erstellen. Einige Probleme, die wegen fehlender Informationen während der Entwicklung am Backend teilweise aufgetreten sind, hätten dadurch eher verhindert werden können. Darüber hinaus ist bei einer Fortsetzung des Projekts sinnvoll, ein Sequenzdiagramm anzulegen. Allgemein lässt sich sagen, dass es sich bei der Entwicklung des technischen Entwurfs um einen komplexen Prozess handelt. Es war eine ausführliche Zusammenarbeit der verschiedenen Teammitglieder notwendig, da es verschiedene Möglichkeiten gibt, ein Softwaresystem zu modellieren, wobei nicht jede Lösung die in Kapitel \ref{chapter:technischer_entwurf} genannten Ziele Skalierbarkeit, Effizienz und Anpassbarkeit erfüllt.
		
	\section[Bewertung der Organisation]{Bewertung der Organisation{\hfill \normalsize Fabio Westphal}}
		Die Herausforderungen einer etwas größeren Gruppe wurden auch bei den gewählten Tools zur Organisation deutlich. Das Zusammenführen der aktuellsten Version der Seminararbeit warf zum Beispiel Probleme auf, wenn mehrere Personen an einem Kapitel arbeiteten. Github konnte die verschiedenen Commits teilweise nicht korrekt zusammenführen, da es sich bei LaTeX nicht um eine typische zeilenbasierte Sprache handelt. So mussten die neuen oder überarbeiteten Texte alle an eine Person geschickt werden, welche diese dann zusammensetzte. Beim nächsten Mal müsste im Vorhinein eine besser geeignete Software für kollaboratives Schreiben gewählt werden. Außerdem kommt es bei mehreren Autoren schnell zu Redundanzen. Bestimmte projektspezifische Sachverhalte werden zwar in mehreren Kapiteln behandelt, müssen aber für den Leser nur einmal ausführlich erklärt werden. Letzlich musste die Arbeit am Ende nochmal in Gänze gelesen werde, um solche inhaltlichen Dopplungen zu minimieren. Vermeiden lässt sich das mit angemessem Aufwand wohl nicht.
		Wie im Kapitel \ref{chapter:SCRUM} bereits erläutert wurden die Vorgaben von SCRUM nicht komplett umgesetzt. Bei täglicher Zusammenarbeit an dem Projekt wäre eine stärkere Orientierung am SCRUM-Ablauf definitiv möglich und nötig gewesen. Zudem wurde die Rolle des Product-Owners nicht im eigentlichen Sinne ausgefüllt, da beispielsweise der Backlog nicht von ihm, sondern von allen Teammitgliedern gemeinsam aufgestellt wurde. Bezüglich des Iterativen Modells lässt sich fesstellen, dass aufgrund des kurzen Entwicklungszeitraums relativ wenig Kundenfeedback eingeholt wurde, um die einzelnen Schritte zu validieren. Dadurch haben wahrscheinlich weniger Rücksprünge stattgefunden als in einem üblichen Projekt. 
		
		Die zeitlich Komponente wurde streckenweise etwas unterschätzt, hier müssten andere Vorlesungen und Prüfungsleistungen mit mehr Puffer einkalkuliert werden. Insgesamt wurde die Projektplanung aber ebenso gut umgesetzt und wie die Kommunikation innerhalb des Teams, da regelmäßige Absprachen und dauerhafte Kommunikationskanäle zu einer klaren Aufgabenverteilung geführt haben.
		
	
	\section[Lessons Learned]{Lessons Learned{\hfill \normalsize Milena Zahn}} \label{Ausblick} 
	% Milena zu Projektumsetzung 
	% Fehlt LaTex Zusammenführung (inhaltlich sowie formell)
	Die Bereiche, an denen gearbeitet wurde, waren vielfältig. Zunächst wurde an der Analyse und der Erarbeitung des Entwurfs gearbeitet. Danach musste das Team sich erarbeiten, wie ein solches Projekt umgesetzt werden kann und was die ersten Schritte für die Umsetzung einer eigenständigen Anwendung sind. Dies war für die meisten Teammitglieder eine komplett neue Erfahrung. Ebenfalls die Verknüpfung des Backends mit dem Frontend war eine herausfordernde Aufgabe, die jedoch durch die unterschiedlichen Kompetenzen des Teams gemeistert werden konnte. 
	
	Neben dem technischen Wissen haben die Teammitglieder auch weitere Kompetenzen in dem Projekt erworben. Eine davon ist, dass es sehr wichtig ist, andere Projekte, an denen die Teammitglieder außerhalb des Moduls arbeiten müssen, bei dem Projektplans zu berücksichtigen. Obwohl dies sehr schwierig ist, kann dies helfen den Projektplan genauer und realistischer zu erstellen. Es ist sehr wichtig für jede Aufgabe genügend Zeit einzuplanen, um unerwartete Probleme zu bewerkstelligen. Die selbständige Erarbeitung von Vorgehensweisen und Lösungskonzepte ist eine weitere in diesem Projekt erworbene und essentielle Kompetenz.
	
	Eine weitere Hürde war die Gruppengröße von acht Personen. Es ist eine große Herausforderung, das Potenzial der Gruppe voll ausschöpfen, weil die Arbeit innerhalb des Teams gut organisiert werden muss. 
	Wie schon das Zitat von Trocholsky aus der Einleitung andeutet, kann Überorganisation dazu führen, dass zu viele Teammitglieder - zumindest über eine zu lange Zeit - mit der Koordination beschäftigt sind und letztlich nur ein Bruchteil der Ressourcen für die eigentliche zu erledigende Arbeit aufgewandt werden kann.
	Um eine übermäßige Koordination der Mitglieder und lange Kommunikationswege zu vermeiden, wurde die Gruppe in Untergruppen mit verschiedenen Aufgabenbereichen aufgeteilt. Somit konnten schnelle Entscheidungsfindungen und kurzfristige Absprachen garantiert werden. 
	
	Die Schwierigkeiten der Gruppengröße wiederspiegelte sich zudem in der Erstellung und vor allem der Zusammenführung der Seminararbeit. Eine gute Struktur und eine klare Aufgabenverteilung sind hierbei sehr wichtig. Die Zusammenführung und das Korrekturlesen der Arbeit nahm in Anbetracht der Anzahl der Projektbeteiligten mehr Zeit in Anspruch als zunächst eingeplant.
	Insgesamt wurden in der Projektzeit viele unterschiedliche, sowohl technische als auch methodische, Kompetenzen erworben und die theoretischen Inhalte der Vorlesungen des zweiten Semesters praktisch angewandt.
	
	
	
	
%% !TEX root =  master.tex
\chapter{Ausblick}
	Hätte, hätte, Fahrradkette.

	Fragt man sich immer woran hats gelegen


%	Literaturverzeichnis
\ihead{} % Neue Header-Definition
\printbibliography[title=Literaturverzeichnis]
\cleardoublepage

% Der Anhang beginnt hier - jedes Kapitel wird alphabetisch aufgezählt. (Anhang A, B usw.)
\appendix
\ihead{\appendixname~\thechapter} % Neue Header-Definition

% appendix.tex einziehen
\chapter{Dynamisches Modell}

\section{Ausführliches Aktivitätsdiagramm Reservierung}

\begin{figure}[H]
	\centering
	\subfigure[Teil 1]{\includegraphics[width=14cm]{img/adReservierung1.pdf}} 
\end{figure}

\begin{figure}[H]
	\ContinuedFloat 
	\centering
	\subfigure[Teil 2]{\includegraphics[width=14cm]{img/adReservierung2.pdf}} 
\end{figure}

\begin{figure}[H]
	\ContinuedFloat 
	\centering
	\subfigure[Teil 3]{\includegraphics[width=14cm]{img/adReservierung3.pdf}} 
	\caption[Ausführliches Aktivitätsdiagramm des Reservierungsprozesses]{\label{fig:adResAusführlich}Ausführliches Aktivitätsdiagramm \\des Reservierungsprozesses}
\end{figure}

% Ehrenwörtliche Erklärung ewerkl.tex einziehen
\input{ewerkl.tex}

\end{document}
